
\newpage \addsec{Newtonsche Mechanik $(0,0,0,0)$}
\label{sec:0000}

Man stelle sich eine Welt vor, in der die Schwerkraft vernachlässigbar ist. In der sich Dinge im Vergleich zur Lichtgeschwindigkeit sehr langsam bewegen. In der Quanteneffekte zu klein sind, um wahrgenommen werden zu können. Und in der die Anzahl der Teilchen sehr begrenzt ist. Eine solche Welt wäre vollständig durch Galileis Raum-Zeitmodell und die Newtonsche Mechanik beschrieben.

Im Hexadekachor-Modell entspricht dies der Theorie, in der die Gravitationskonstante $G$, der Kehrwert der Lichtgeschwindigkeit $c^{–1}$, das Plancksche Wirkungsquantum $h$ und die Boltzmann-Konstante $\kboltz$ allesamt verschwinden: Dies ist das kleine Tetraeder im Zentrum des Modells. Es symbolisiert den Ursprung der modernen Physik sowohl in konzeptioneller als auch in historischer Hinsicht. Alle Gegenstände in dieser Welt haben einen wohldefinierten Ort im absoluten Raum. Die Zeit ist ein Parameter, der überall gleich und unentwegt das Verrinnen von Sekunden, Minuten und Stunden zählt. Zu jeder Bewegung gehört ein fester, berechenbarer Weg. Und jede Beziehung zwischen den Teilchen ist festgelegt. Klingt das vertraut? Wie wir alle wissen, beschreibt diese Theorie nicht die Realität. Sie ist lediglich eine erste, aber kontrollierte Annäherung an die Vielfalt der physischen Welt um uns herum.


\subsection*{Newtons Grundlagen}

Aus der Sicht der modernen Physik beschäftigt sich die Newtonsche Mechanik mit der Bewegung hinreichend langsamer klassischer Körper unter dem Einfluss eines Systems beliebiger physikalischer Kräfte. Daher kennt die reine Newtonsche Mechanik \textit{per se} keine Schwerkraft, kein Licht, kein Wirkungsquantum, und es ist ihr egal, ob es nur wenige oder sehr viele Teilchen gibt. Sie ist die Theorie $(0,0,0,0)$ des Hexadekachor-Modells und definiert den Ursprung eines kartesischen Koordinatensystems, dessen Achsen durch vier grundlegende Naturkonstanten definiert sind. Gleichzeitig weist dieses System allen anderen Theorien ihren richtigen Platz in diesem Modell zu.

Newton fasste seine Erkenntnisse in drei Gesetzen zusammen. Das zweite Gesetz ist wohl das wichtigste und besagt, dass die auf einen Körper ausgeübte Kraft $\vec{F}$ gleich der Masse $m$ dieses Körpers mal seiner Beschleunigung $\vec{a}$ ist:

\begin{equation*}
  \vec{F} =m\, \vec{a}\,.
\end{equation*}
%
In dieser Formel treten eindeutig keine fundamentalen Konstanten auf.

Wenn man die Gravitation berücksichtigt, bleibt die Newtonsche Mechanik intakt. Sie wird lediglich zur Theorie $(G,0,0,0)$ erweitert, indem man Newtons Gravitationsgesetz und die grundlegende Idee der Identität von träger und schwerer Masse hinzufügt. Damit definiert diese Theorie ein neues Tetraeder des Modells.


\subsection*{Galileis Raum-Zeitmodell}

Die Newtonsche Mechanik basiert auf Galileis Modell von Raum und Zeit. Sie geht davon aus, dass zwei Inertialsysteme physikalisch ununterscheidbar sind. In einem berühmten Gedankenexperiment hat Galilei ein Schiff betrachtet, das sich geradlinig mit konstanter Geschwindigkeit bewegt. Unter Deck, alle Luken dicht und kein Mobiltelefon zur Hand, kann man zwar die eigene Bewegung bezogen auf das Schiff bestimmen, nicht aber bezogen auf die See. Dieses Gedankenexperiment war für Galilei von solcher Strahlkraft, dass er es im Frontispiz seines \enquote{Dialogs über die zwei Weltsysteme} prominent zwischen den Figuren von Aristoteles, Ptolemäus und Kopernikus verewigte. Darüber hinaus macht die Newtonsche Mechanik eine absolute Zeit geltend, die in sämtlichen Inertialsystemen gleich verläuft. Die Geschwindigkeiten von Inertialsystemen addieren sich linear. Damit ist es möglich, alltägliche mechanische Systeme hinreichend genau zu beschreiben.


\subsection*{Nachträgliche Korrekturen}

Galileis Modell von Raum und Zeit führt jedoch zu logischen Widersprüchen mit den Maxwell-Gleichungen der Elektrodynamik. Daher sind für große Geschwindigkeiten grundlegende Korrekturen notwendig, die nur im Rahmen von Einsteins spezieller Relativitätstheorie verständlich sind, also in der Theorie $(0,c^{-1},0,0)$.

Noch drastischere Korrekturen erfährt die klassische Mechanik auf atomaren und subatomaren Längenskalen. Sie verliert ihren deterministischen Charakter und wurde im ersten Drittel des 20.\ Jahrhunderts im Rahmen der Quantenmechanik $(0,0,h,0)$ als Theorie von Wahrscheinlichkeitsamplituden neu formuliert.

Schon vorher hatten Maxwell und Boltzmann und viele andere gezeigt, dass die Thermodynamik als statistische Theorie sehr vieler klassischer, Newtonscher Teilchen verstanden werden kann, als klassische statistische Mechanik $(0,0,0,\kboltz)$.

Sämtliche Korrekturen zur klassischen Mechanik verweisen also auf neue Tetraeder des Modells. Die beiden Theorien $(0,0,0,0)$, die Newtonsche Mechanik, und $(G,0,0,0)$, die Newtonsche Gravitationstheorie, wurden von Newton zeitgleich entwickelt; hier widerspricht also die Systematik des Hexadekachors in gewisser Weise der Geschichte der Physik. Newtons Theorie wird daher als zwei Tetraeder des Modells visualisiert, aber er selbst hat diese Unterscheidung nie getroffen. Auf der anderen Seite wurden die Korrekturen $(0,c^{-1},0,0)$, $(0,0,h,0)$ und $(0,0,0,\kboltz)$ bekanntermaßen erst zu Beginn des 20.\ Jahrhundert eingeführt. Hier spiegelt das Modell die Wissensgeschichte wider.
