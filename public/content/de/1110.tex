
\newpage \addsec{TOE: Theorie von Allem $(G,c^{-1},h,0)$}
\label{sec:1110}

Der bisher unentdeckte heilige Gral der modernen Physik. Diese Theorie würde die drei fundamentalen Konstanten $G,c^{−1}$ und $h$ umfassen, unter Ausschluss Plancks vierter Wahl $\kboltz$, der Boltzmann-Konstante. Es gibt eine ganze Reihe ambitionierter Anwärter auf den Thron, allen voran die Stringtheorie und die Schleifenquantengravitation. Allerdings ließ sich bisher für keine dieser beiden Theorien überprüfen, ob sie korrekt ist~-- vermutlich sind am Ende sogar beide falsch, zumindest aber unvollständig. Durch die Kombination der drei fundamentalen Naturkonstanten $G$, $c^{-1}$ und $h$ können wir die natürlichen Einheiten für Raum, Zeit und Masse ableiten, die sogenannten Planck-Einheiten~-- wobei die Temperatur außer Acht gelassen wird. Dabei stellt sich heraus, dass die Quantengravitation in Größenordnungen relevant wird, die diesen Planck-Einheiten entsprechen.

Leider sind diese Skalen extrem weit von allem entfernt, was wir experimentell direkt erschließen können. So liegt die Planck-Länge bei etwa $10^{−35}$ Metern, das ist eine geradezu unvorstellbare $10^{20}$-mal kleinere Skala als die Größe des Protons.


\subsection*{Eine Weltformel?}

Was ist nun die Theorie $(G,c^{-1},h,0)$ des Hexadekachors der Physik? Sie sollte mathematisch konsistent und falsifizierbar sein, also experimentell überprüfbare Vorhersagen liefern können. Im Deutschen wird diese gesuchte Theorie oft verkürzt als \emph{Weltformel} (\emph{world formula}) bezeichnet, obwohl es natürlich \textit{a priori} unklar ist, ob sie im Falle ihrer Existenz auf einer einzigen Formel basiert. Im Englischen heißt sie dagegen ganz unbescheiden \emph{theory of everything} (TOE), übersetzt also \emph{Theorie von Allem}.

Die Theorie von Allem sollte insbesondere Quantenfeldtheorie und allgemeine Relativitätstheorie vereinen. Die Gravitation soll also \emph{quantisiert} und die Konstante $h$ mit der Gravitation in Verbindung gebracht werden. Ebenso sollte $G$ mit der Quantenfeldtheorie verbunden werden. \emph{Quantisiert} man jedoch die allgemeine Relativitätstheorie in Analogie zur Elektrodynamik auf naive Weise, z.\,B.\ indem man über allen möglichen Metriken summiert, führt dies zu einer Katastrophe: Eine unendliche Vielzahl von unbezähmbaren \emph{Divergenzen} zerstört die Voraussagekraft der Theorie.

Die beiden prominentesten aller aktuell verfolgten Ansätze, die diese Schwierigkeiten zu umgehen suchen, sind die Superstringtheorie und die Schleifenquantengravitation. Beiden ist gemein, dass sie von einer experimentellen Überprüfbarkeit weit entfernt sind, und, damit zusammenhängend, dass es jeweils eine ungeheure Vielzahl unterschiedlicher Versionen gibt. Es handelt sich also eher um Theorieklassen als um eindeutig definierte Theorien.


\subsection*{Superstringtheorie}

Die Superstringtheorie hat ihre Wurzeln in der Quantenfeldtheorie. Deren Punktteilchen werden durch schwingende Saiten, die sogenannten Strings, ersetzt, um die verschiedenen Teilchen der Natur als Schwingungszustände der Strings darstellen zu können. Durch diesen Ansatz verschwinden die genannten Divergenzen, und Teilchen- und Gravitationsphysik werden vereint. Ein hypothetisches \emph{Graviton} tritt auf, alle Welten werden zehn- oder sogar elfdimensional, und unsere eigene Welt ist offenbar nur eine von mindestens $10^{500}$ vielen. Einige dieser Welten ähneln zumindest dem Standardmodell der Teilchenphysik, aber leider wurde noch keine exakte Übereinstimmung entdeckt. Ein spekulativer Ausweg ist die Idee des Multiversums, in dem viele dieser möglichen Welten in Paralleluniversen existieren, die, etwas deprimierend, nicht miteinander kommunizieren. In diesem Szenario wären die vielen noch ungeklärten Parameter des Standardmodells letztlich entweder zufällig gesetzt oder aber schlichtweg dadurch festgelegt, dass sie intelligentes Leben ermöglichen (anthropisches Prinzip). Nicht so schön.


\subsection*{Schleifenquantengravitation}

Die Schleifenquantengravitation hat ihre Wurzeln in der allgemeinen Relativitätstheorie. Letztere wird direkt quantisiert; dies führt zu einer Art \enquote*{Quantenschaum} auf der Planck-Skala im Bereich von $10^{-35}$ Metern. Es gibt viele sich teilweise widersprechende Zugänge, aber in bisher keinem eine natürliche Vereinigung mit den Quantenfeldtheorien des Standardmodells. Diese müssen im Grunde genommen \enquote*{von Hand} hinzugefügt werden. Auch nicht schön.


\subsection*{Offene Fragen \dots}

Benötigt man überhaupt eine Theorie, die Quantenfeldtheorie und allgemeine Relativitätstheorie vereint? In dieser Frage herrscht weitgehende, bejahende Einigkeit in der Physik. Es ist nämlich im Rahmen der aktuellen physikalischen Theorien offensichtlich, dass sowohl die Detailbeschreibung von Schwarzen Löchern als auch diejenige des Urknalls die Quantenphysik erfordern. Zudem ist es zumindest experimentell klar, dass etwas Grundlegende im aktuellen Theoriegebäude der Physik fehlt: Etwa 96 Prozent der Materie und Energie unseres Universums sind wohl noch unbekannt, oder, wie man sagt, \enquote*{dunkel}.
