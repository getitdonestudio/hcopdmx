
\newpage \addsec{Spezielle Relativitätstheorie und Elektrodynamik $(0,c^{-1},0,0)$}
\label{sec:0100}

In seinem \emph{annus mirabilis} 1905 veröffentlichte Albert Einstein vier berühmte Arbeiten. Im dritten Artikel \enquote{Zur Elektrodynamik bewegter Körper} stellte er die spezielle Relativitätstheorie vor. Diese basiert auf zwei Postulaten: der Invarianz des Inertialsystems~-- die Gesetze der Physik sind in allen Inertialsystemen identisch~-- und der Konstanz der Vakuumlichtgeschwindigkeit $c$, d.\,h., dass diese unabhängig vom Bewegungszustand der Lichtquelle ist. Die These, dass $c$ eine fundamentale Konstante für alle möglichen Beobachter sei, also unabhängig von deren eigenem Bewegungszustand, schien zumindest auf den ersten Blick widersprüchlich und erforderte ein radikales Hinterfragen der Annahmen von Galileo und Newton über die Struktur von Raum und Zeit. Als Ergebnis seiner Analyse erhielt Einstein die vielleicht berühmteste Formel der modernen Physik

\begin{equation*}\label{emc2}
  E=m\,c^2\,,
\end{equation*}
%
die die Äquivalenz von Masse ($m$) und Energie ($E$) ausdrückt.

Der erste Beweis für die Endlichkeit der Lichtgeschwindigkeit wurde 1676 von Ole Rømer erbracht, und die erste grobe Schätzung erfolgte zwei Jahre später durch Christiaan Huygens. Seitdem wurde die Genauigkeit ständig verbessert, bis die Lichtgeschwindigkeit schließlich 1983 festgelegt wurde auf den \emph{exakten} Wert

\begin{equation*}
  c=\SI{299 792 458}{\meter \per \second}\,.
\end{equation*}

Im Hexadekachor-Modell stellt sich der relevante Parameter tatsächlich als die inverse Lichtgeschwindigkeit $c^{-1}$ heraus, da der nicht-relativistische Grenzwert durch die Annahme einer unendlichen Lichtgeschwindigkeit erhalten wird. Dementsprechend erscheint die spezielle Relativitätstheorie als stachelförmiger Tetraeder $(0, c^{-1}, 0, 0)$. Von zentraler Bedeutung im Hexadekachormodell ist, dass dieser Tetraeder auch die Maxwellsche Theorie des klassischen Elektromagnetismus umfasst. 


\subsection*{Elektromagnetismus}

Auch wenn erst mit der speziellen Relativitätstheorie die Lichtgeschwindigkeit als neue fundamentale Konstante der Physik verankert wurde, reichen ihre Anfänge bis ins 19.\ Jahrhundert zurück. 1873 vereinheitlichte James Clerk Maxwell nicht nur Elektrizität und Magnetismus, sondern identifizierte zudem Licht als ein elektromagnetisches Phänomen. Da der Elektromagnetismus als physikalisches Phänomen auf der Grundlage eines alles durchdringenden Äthers erklärt wurde, lag es nahe, dessen Existenz experimentell zu belegen. Die berühmten Interferometer-Experimente aus den 1880er Jahren von Edward Morley und Albert Michelson zeigten allerdings, dass die Ausbreitungsgeschwindigkeit von Licht in keiner Weise von der Bewegung der Erde durch einen Äther abhängt, und lieferten damit die experimentelle Grundlage für Einsteins Postulate. Es begann die komplexe Geschichte der Lorentz-Transformationen, die das exakte Transformationsverhalten der Maxwell-Gleichungen beschreiben.


\subsection*{Lorentz-Transformationen}

Die beiden Postulate der speziellen Relativitätstheorie sind nur dann miteinander vereinbar, wenn man die Lorentz-Transformationen verwendet, um zwischen den Koordinaten zweier Inertialsysteme zu wechseln, die sich mit konstanter Geschwindigkeit relativ zueinander bewegen. Die Galilei-Transformationen ergeben sich dann als Näherung für niedrige relative Geschwindigkeiten der Bezugssysteme. Darüber hinaus spiegeln sich die Konsequenzen für Bezugssysteme, die sich mit Geschwindigkeiten nahe der  Lichtgeschwindigkeit relativ zueinander bewegen, in zunächst rätselhaft erscheinenden Effekten wie der Zeitdilatation oder der Längenkontraktion wider.


\subsection*{Minkowski-Raumzeit}

1907 zog Hermann Minkowski die mathematischen und konzeptionellen Konsequenzen aus Einsteins Ideen und schlug auf revolutionäre Weise vor, den dreidimensionalen euklidischen Raum durch eine vierdimensionale Raumzeit zu ersetzen. Diese mathematisch präzise Verschmelzung von Raum und Zeit brachte die Physik auch in Kontakt zu philosophischen Konzepten wie der kosmologischen Ideen der Inkas oder der taoistischen Sichtweise von Raum und Zeit als \enquote*{komplexes kosmisches Netz}. Dies hinderte den Berliner Senat leider dennoch nicht daran, den 1994 verliehenen Ehrengrabstatus des Urnengrabs Minkowskis auf dem Berliner Friedhof Heerstraße 2014 wieder zu entziehen.

In Anlehnung an Minkowskis Formalismus können wir den klassischen dreidimensionalen Impuls $\mathbf{p}$ zu einem Vierervektor verallgemeinern, dessen Zeitkomponente die Energie $E$ angibt. In der Minkowski-Raumzeit wird der Energie-Impuls-Vierervektor als $P = (E/c, \mathbf{p})$ ausgedrückt, mit der nicht-euklidischen quadrierten Norm $P^2=(E/c)^2 - \mathbf{p}^2 = m^2c^2$. Für den Spezialfall eines ruhenden Objekts, $\mathbf{p} = 0$, ergibt sich dann tatsächlich $E=m c^2$.

Wenn man andererseits $m=0$ setzt, erhält man die Beziehung zwischen der Energie und dem Impuls masseloser Teilchen wie Photonen, den Quanten von Licht und Strahlung:

\begin{equation*}\label{emc2}
  E=c\, |\mathbf{p}|\,.
\end{equation*}
%
Trotzdem weiß die spezielle Relativitätstheorie $(0,c^{-1},0,0)$ scheinbar nichts über die Theorie der Quantenmechanik $(0,0,h,0)$, die einem völlig anderen stachelförmigen Tetraeder des Hexadekachors entspricht!
