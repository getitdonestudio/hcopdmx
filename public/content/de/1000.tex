
\newpage
\addsec{Newtonsche Gravitationstheorie $(G,0,0,0)$}
\label{sec:1000}

Eine der größten Leistungen Newtons bestand in der Entdeckung seines allgemeinen Gesetzes der Gravitation. In moderner Formulierung besagt es, dass jeder Massenpunkt im Universum jeden anderen Massenpunkt \emph{augenblicklich} mit einer Kraft $F$ anzieht, die proportional zum Produkt ihrer Massen $m$ und $M$ und umgekehrt proportional zum Quadrat ihres Abstandes $r$ ist:

\begin{equation*}\label{eq:gravitation}
  F=G\,\frac{Mm}{r^2}\,.
\end{equation*}
%
Die zugehörige Proportionalitätskonstante $G$ wurde ihm zu Ehren viel später als Newtonsche Gravitationskonstante bezeichnet. Experimentell wurde $G$ erstmals, unbeabsichtigt und indirekt, 1798 von Henry Cavendish bestimmt, als er sein berühmtes Experiment zur Messung der Dichte der Erde durchführte. Der aktuelle Wert beträgt

\begin{equation*}
  G = \SI{6.67430(15)d-11}{\cubic\metre\per\kilogram\per\square\second}\,,
\end{equation*}
%
mit einer überraschend großen relativen Unsicherheit von \num{2,2 d-5}. Als solche ist sie die einzige der vier Fundamentalkonstanten des Hexadekachors $G$, $c^{-1}$, $h$ und $\kboltz$, die noch nicht auf einen exakten Wert festgelegt wurde, und deren Genauigkeit man mit verschiedenen Experimenten fortlaufend weiter zu erhöhen versucht.

Im Hexadekachor-Modell entspricht die Newtonsche Gravitation dem stachelförmigen Tetraeder $(G,0,0,0)$, der nach unten in Richtung Boden zeigt. In Kombination mit der klassischen Newtonschen Mechanik, d.\,h.\ dem innersten Tetraeder $(0,0,0,0)$, kann diese Theorie verwendet werden, um sowohl das Herunterfallen von Objekten auf der Erde als auch das \enquote*{Herumfallen} von Himmelskörpern wie Monden, Planeten, Meteoriten und Satelliten im Planetensystem mit erstaunlicher Genauigkeit zu beschreiben.


\subsection*{Newtons Gravitationstheorie}

Im dritten Buch \enquote{De mundi systemate} seiner \enquote{Philosophiæ Naturalis Principia Mathematica} führt Isaac Newton die Bewegung der Planeten und ihrer Satelliten auf die Wirkung einer universellen Gravitation zurück. Newton zufolge hält die resultierende Kraft alle Himmelskörper in ihrer Bahn. Sie verhält sich umgekehrt proportional zum Quadrat des Abstandes zwischen den Massenschwerpunkten zweier Objekte mit einer sphärisch symmetrischen Massenverteilung:

\begin{quote}
  Wenn die Materie zweier Kugeln, die sich gegenseitig anziehen, in den Bereichen, die sich in gleicher Entfernung vom Zentrum befinden, homogen ist, verhält sich das Gewicht jeder Kugel umgekehrt proportional zum Quadrat des Abstands zwischen ihren Zentren. (Newton 1687)
\end{quote}


\subsection*{Äquivalenzprinzip}

Ein integraler Bestandteil von Newtons Gravitationstheorie ist das schwache Äquivalenzprinzip. Dieses besagt, dass träge und schwere Masse eines jeden Körpers identisch sind. Bereits Galileo hatte experimentell nachgewiesen, dass zwei Objekte mit \emph{ungleicher} Masse dennoch gleich schnell fallen, jedenfalls im Prinzip. Den notwendigen Beweis für diese Äquivalenz lieferte Newton experimentell mit einem speziellen Paar von Pendeln. Bis heute wird diese Äquivalenz mit äußerst komplexen Experimenten untersucht. Eine Abweichung zwischen schwerer und träger Masse hat sich dabei noch nie offenbart. Besonders eindrucksvoll demonstrierte die Mondmission Apollo 15, wie eine Feder und ein Hammer, die gleichzeitig aus derselben Höhe fallen gelassen wurden, gleichzeitig auf dem Boden auftrafen.


\subsection*{Fernwirkung und Einsteinsche Gravitation}

Für Newton besaß das Gravitationsgesetz ausschließlich relationalen Charakter; er führte die Gravitationskonstante $G$ nicht ein. Doch auch aus der Proportionalität der universellen Gravitation leitete er epochale Konsequenzen ab~-- zuvorderst, dass sich jeder Planet auf einer elliptischen Umlaufbahn um die Sonne bewegt, die sich in einem der Brennpunkte der Ellipse befindet. Newton formulierte damit eine dynamische und mathematische Herleitung für die bereits empirisch bekannte Bahn von Planeten: \enquote{whereas Kepler guessed right at the Ellipsis} (Newton an Halley 1686). Trotz der revolutionären Erfolge blieb für viele Zeitgenossen und Nachfolger Newtons bis hin zu Einstein der Charakter der instantanen Fernwirkung in seinem Gravitationsgesetz äußerst rätselhaft und fragwürdig. Letzterer löste das Rätsel nach der Entwicklung seiner Allgemeinen Relativitätstheorie, siehe $(G,c^{-1},0,0)$, die auch als Einsteinsche Gravitationstheorie bekannt ist.