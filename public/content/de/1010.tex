
\newpage \addsec{Nichtrelativistische Quantengravitation $(G,0,h,0)$}
\label{sec:1010}

Das Hinzuschalten der Gravitationskonstante $G$ zur nicht-relativistischen Quantenmechanik gestaltet sich zunächst denkbar unproblematisch: Man löst einfach die Schrödinger-Gleichung für massive Teilchen, die sich im Newtonschen Gravitationspotential bewegen. Interessanter wird es, wenn man die Existenz einer noch zu etablierenden \emph{theory of really everything} (TORE), also einer \emph{Theorie von wirklich Allem}, 
annimmt und sich dann vorstellt, darin die inverse Lichtgeschwindigkeit $c^{−1}$ sorgfältig abzuschalten, indem man in allen Formeln dieser hypothetischen Theorie mathematisch den Grenzwert $c\rightarrow\infty$ bildet.

Führt dies wirklich nicht zu messbaren Effekten? Haben wir wirklich nur das eher langweilige quantenmechanische Szenario, das eingangs skizziert wurde? Die Antwort steht noch aus. Insofern stellt die nicht-relativistische Quantengravitation gegenwärtig keine eigene Theorie der Physik dar und wird auch nur relativ wenig beforscht. Es sollte aber nicht unerwähnt bleiben, dass so mancher unkonventionelle Ansatz zur Quantengravitation mit genau dieser Möglichkeit der Brechung der Lorentz-Symmetrie zwischen Raum und Zeit spielt.


\subsection*{Experimentelle Befunde}

Seit den 1960er Jahren wurden Experimente mit Quantenteilchen in klassischen Gravitationsfeldern, und insbesondere dem Feld der Erde, durchgeführt. 
Offensichtlich ist es wichtig, langsame Quantenteilchen zu verwenden, wofür sich Neutronen besonders anbieten. So führten beispielsweise Colella, Overhauser und Werner 1975 ein Experiment durch, dessen Resultat sowohl von der Newtons Gravitationstheorie als auch von der Quantenmechanik abhing. Ein Strahl sehr langsamer Neutronen wurde zunächst aufgespalten und anschließend interferometrisch untersucht. Durch Rotation der Messvorrichtung um einen Winkel $\Phi$ konnten sie zeigen, dass eine quantenmechanische Phasenverschiebung der langsamen Neutronen durch deren Wechselwirkung mit dem Gravitationsfeld der Erde auftritt. Der experimentelle Befund stimmte jedoch bestens mit der theoretischen Vorhersage der Schrödinger-Gleichung mit Newtonschem Gravitationspotential überein, so dass sich (leider!) keine wirklich neuen Einsichten ergaben. Neuere Experimenten konnte sogar die Quantisierung schwacher Bindungszustände ultra-kalter, d.\,h.\ sehr langsamer Neutronen, im Gravitationsfeld der Erde nachweisen.


\subsection*{Weitere Hinweise}

Diese Art von Experiment ist möglicherweise relevant, um verschiedene Szenarien für die Theorie von Allem (TOE) zu unterstützen oder auszuschließen. Beispielsweise sagen einige Versionen der Superstringtheorie Abweichungen vom Newtonschen Gravitationspotenzial auf Längenskalen weit überhalb der Planck-Skala voraus. Allerdings konnten derartige Abweichungen bisher leider noch nicht gemessen werden.

Auch rein theoretisch inspiriert die durch den $c^{−1}\rightarrow 0$ Limes erzeugte Brechung der Lorentz-Symmetrie zwischen Raum und Zeit immer wieder die Forschung. So schlug beispielsweise Petr Hořava 2009 eine neue Gravitationstheorie vor, die die allgemeine Relativitätstheorie auf kleinen Längenskalen modifiziert. Dies hat bereits zu mehr als \num{2600} nachfolgenden Veröffentlichungen geführt. Dennoch ist der Status dieser sogenannten Hořava-Lifshitz Gravitationstheorie nach wie vor unklar.
