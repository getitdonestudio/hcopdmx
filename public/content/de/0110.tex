
\newpage \addsec{Quantenfeldtheorie $(0,c^{-1},h,0)$}
\label{sec:0110}

Die Quantenmechanik wurde in den 1920er Jahren zunächst im nicht-relativistischen Grenzfall $c^{−1}=0$ formuliert, obwohl die Richtigkeit von Einsteins spezieller Relativitätstheorie bereits seit Jahren allgemein anerkannt war, mit der bemerkenswerten Ausnahme einiger antisemitischer Idioten. Dirac gelang es 1928, eine eigenartige relativistische Wellengleichung aufzustellen, die die Schrödinger-Gleichung auf höchst nicht-triviale Weise verallgemeinerte. Sie ist heute direkt vor Isaac Newtons Grab in der Westminster Abbey in der eleganten Form $i \gamma \cdot \partial\, \psi = m\, \psi$ eingraviert, wobei $c^{−1}$ und $\hbar=\frac{h}{2 \pi}$ auf eins gesetzt wurden. Wenn man sie wieder einsetzt, lautet die Dirac-Gleichung

\begin{equation*}
  i\, \hbar\,\, \gamma \cdot \partial\, \psi = m\,c\, \psi\,.
\end{equation*}
%
Insbesondere erklärte sie den Spin des Elektrons und sagte dessen Antiteilchen voraus. Ein Triumph war die experimentelle Bestätigung dieses Positrons durch Carl Anderson im Jahr 1932.

Allerdings führte die Interpretation der Dirac-Gleichung als Wellengleichung zu rätselhaften Widersprüchen. Zudem ergaben sich Probleme bei der Quantisierung der Elektrodynamik, bei der ebenfalls $c^{-1}$ eingeschaltet ist. Als Ausweg wurde die Quantenfeldtheorie (QFT) in den späten 1940er Jahre als eine relativistisch konsistente Vielteilchentheorie entwickelt. Diese führte etwa 20 Jahre später zum Standardmodell der Elementarteilchen, das seitdem alle Naturkräfte mit Ausnahme der Gravitation präzise beschreibt.


\subsection*{Standardmodell}

Die korrekte Quantisierung elektromagnetischer Felder~-- genauer gesagt der sogenannten abelschen Eichfelder~-- führt zur Quantenelektrodynamik (QED). Sie ist die quantitativ am genauesten überprüfte physikalische Theorie, die jemals aufgestellt wurde. Die Quanten dieser Felder sind nichts anderes als Einsteins Photonen, die ebenfalls einen Spin besitzen. Im Gegensatz zum Elektron ist deren Wert eins. Und auch die Kernkräfte beschreibt man sehr erfolgreich mit Quantenfeldtheorien, wobei weitere Spin-1-Teilchen zum Einsatz kommen: die acht Gluonen der starken Kernkräfte und die W$^+$, W$^-$- und Z-Bosonen der schwachen Kernkräfte. Um diese Teilchen und die ihnen entsprechenden Naturkräfte zu beschreiben, muss man die abelschen Eichfelder durch mathematisch deutlich komplexere nichtabelsche Felder ersetzen. Insgesamt ergibt sich daraus das aktuelle Standardmodell der Elementarteilchenphysik.

Seine nichtabelsche Eichsymmetriegruppe ist $\mathrm{SU}(3) \times \mathrm{SU}(2) \times \mathrm{U}(1)$, wobei der erste Faktor mit den starken Kernkräften (Quantenchromodynamik: QCD), der zweite mit den schwachen Kernkräften und der dritte mit der QED zusammenhängt. Ein offenes Rätsel ist, warum die Natur genau diese Eichgruppe wählt. Ein weiteres Rätsel sind die ungeklärten 19 freien Parameter des Standardmodells, das um weitere sieben freie Parameter für die schwer fassbaren, leicht massiven und chamäleonartigen Neutrinos erweitert wurde.

Besonders hervorzuheben ist das Higgs-Feld des Standardmodells, das bereits in den frühen 1960er Jahren vorhergesagt und 2012 nach langer Suche endlich mit dem Large Hadron Collider (LHC) am CERN in Genf entdeckt wurde. Es verleiht den meisten Elementarteilchen eine Masse, siehe das $m$ in der obigen Dirac-Gleichung im Fall der Spin $\frac12$ Teilchen.


\subsection*{Das Graviton}

Bisher nicht gelungen ist es, die Gravitationstheorie in das Standardmodell zu integrieren. Das dazugehörige hypothetische Teilchen, das die Gravitationskraft trägt, wurde allerdings schon getauft. Es heißt Graviton und müsste, wenn es existiert, den Spin 2 haben. Seine Einbeziehung würde vom Tetraeder $(0, c^{-1},h,0)$ zum Tetraeder $(G,c^{-1},h,0)$ des Hexadekachors führen, d.\,h.\ zur Theorie von allem (TOE).


\subsection*{Quantenfeldtheorie 2.0}

Als Theorie scheint die QFT trotz der großen Anzahl an Lehrbüchern, die darüber geschrieben wurden, bisher nicht wohldefiniert zu sein. Für einen Mathematiker ist sie daher wenig befriedigend. Der wichtigste praktische Ansatz, mit diesem Problem umzugehen, sind ausgefeilte Störungsmethoden, die auf Feynman-Diagrammen basieren, oder eine numerische, aber nicht-störende Neuformulierung, die als Gitter-Eichtheorie bezeichnet wird. Es gibt spannende Versuche, völlig neue Ansätze für die QFT zu entwickeln. Zudem gibt es faszinierende aktuelle Anwendungen des Formalismus der QFT in der Gravitationswellenforschung in der klassischen $(G,c^{-1},0,0)$-Theorie (= allgemeine Relativitätstheorie).
