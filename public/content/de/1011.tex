
\newpage \addsec{Nicht-relativistische Quantengravitation und Temperatur $(G,0,h,\kboltz)$}
\label{sec:1011}

Zum jetzigen Zeitpunkt existiert noch keine allgemein anerkannte Theorie der Quantengravitation, auch wenn es experimentell nicht verifizierte und theoretisch unvollständige Kandidaten wie die Stringtheorie oder die Schleifenquantengravitation gibt. Diese Situation verbessert sich natürlich nicht durch die Einbeziehung der Temperatur, wenngleich es interessante Vermutungen über die Konsequenzen einer solchen allumfassenden \emph{theory of really everything} (TORE) gibt, also einer \emph{Theorie von wirklich Allem}, die alle vier Konstanten $(G, c^{-1}, h, \kboltz)$ mit einbezieht, wie etwa die Hawking-Strahlung oder die Bekenstein-Hawking-Entropie.

Man könnte vermuten, dass eine nicht-relativistische (d.\,h.\ $c^{-1}=0$) Version des gesuchten TORE-Modells, basierend auf den Konstanten $(G,0,h,\kboltz)$, leichter zu finden sein sollte. Dies scheint jedoch nicht der Fall zu sein, und es wurden im Grunde bisher keine wirklich brauchbaren Vorschläge gemacht. Allerdings gibt sehr interessante, aktuelle Experimente zu Quantenteilchen, Temperatur und Schwerkraft: Dabei handelt es sich um Studien zu ultra-kalten Neutronen (UCN), die auf frühe Ideen von Enrico Fermi aus der Mitte der 1930er Jahre zurückgehen.

UCN sind im Wesentlichen freie und extrem langsame (= kalte) Neutronen, die in einem geeigneten Behälter eingeschlossen werden. Dessen Wände reflektieren vollständig alle darauf aufprallenden Neutronen. Dadurch können UCN als eine Art verdünntes ideales Gas mit einer Temperatur von weniger als \qty{4}{\milli\kelvin} (Millikelvin) behandelt werden. Da die kinetische Energie der UCN sehr gering ist, spielt die Gravitationskraft eine entscheidende Rolle bei der Beschreibung des Systems. Diese Experimente werden als \enquote*{Neutronen in Flaschen} bezeichnet.

Es gibt ein interessantes Rätsel im Zusammenhang mit der Lebensdauer freier Neutronen, wenn man die in den Flaschenexperimenten gemessenen Werte mit denjenigen vergleicht, die in Neutronenstrahlexperimenten gemessen wurden. Die Lebensdauer, die aktuell (2024) von der Particle Data Group angegeben wird, beträgt \qty{878.4(0.5)}{\second} (ungefähr 15 Minuten). Dies liegt nahe an den Werten, die durch die Flaschenexperimente ermittelt wurden, aber etwa \qty{9}{\second} von den typischen Werten entfernt, die durch die Strahlenexperimente ermittelt wurden. Es gibt eine intensive Debatte darüber, wie man ein besseres experimentelles, aber vor allem theoretisches Verständnis dieser Diskrepanz erlangen kann.
