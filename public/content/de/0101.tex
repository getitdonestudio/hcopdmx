
\newpage \addsec{Spezielle Relativitätstheorie und statistische Mechanik $(0,c^{-1},0,\kboltz)$}
\label{sec:0101}

Wie ändert sich die Temperatur bei Lorentz-Transformationen? Oder, direkter gefragt: Welche Temperatur hat ein bewegter Körper? Eine verallgemeinerte Theorie der Thermodynamik, die mit den Prinzipien der speziellen Relativitätstheorie korrekt vereint ist, d.\,h.\ eine Theorie der relativistischen Thermodynamik, ist seit ihrem ersten Vorschlag durch Max Planck und seinen Doktoranden Kurd von Mosengeil und unabhängig davon durch Albert Einstein im Jahr 1907 Gegenstand hitziger Debatten. Im Hexadekachor-Modell würde diese Theorie dem Tetraeder entsprechen, das mit $(0,c^{-1},0,\kboltz)$ bezeichnet ist.


\subsection*{Temperaturverwirrungen}

Welche thermodynamischen Größen bleiben bei Lorentz-Transformationen unverändert? Im Falle der Entropie, unter Berufung auf deren statistische Definition~-- also die Anzahl der möglichen mikroskopischen Zustände eines gegebenen makroskopischen Systems~-- sind sich alle Autoren über ihre relativistische Invarianz einig. Bei Temperatur, Druck und den damit verbundenen thermodynamischen Potenzialen wurde jedoch noch kein vollständiger Konsens erzielt. Laut von Mosengeils posthum veröffentlichter Dissertation (er starb tragischerweise im Alter von 22 Jahren bei einer Wanderung in den Alpen) gilt:

\begin{quote}
  Zwei Körper, die der ruhende Beobachter als gleich heiß bezeichnet, [können] einem bewegten Beobachter verschieden heiß erscheinen [\dots], nämlich dann, wenn die Körper verschiedene Geschwindigkeit haben. Am höchsten wird die Temperatur eines Körpers immer dem Beobachter erscheinen, der relativ zu ihm ruht. (Von Mosengeil 1907)
\end{quote}
%
Seiner Meinung nach transformiert sich die Temperatur wie die Strahlungsintensität eines sich bewegenden schwarzen Körpers gemäß

\begin{equation*}\label{reltemplower}
  T = T_0\,\sqrt{1-\frac{v^2}{c^2}}\,,
\end{equation*}
%
wobei $T$ und $T_0$ die Temperaturen im bewegten bzw.\ ruhenden Bezugssystem sind und $v$ die relative Geschwindigkeit im Bezugssystem bezeichnet. Dieses Transformationsgesetz wird zwar in einem Großteil der neueren Literatur akzeptiert, aber eben nicht durchgängig.

Und in der Tat ist folgender Aspekt verwirrend: Wenn, wie oft behauptet wird, die Temperatur einfach eine Form von Energie wäre, die mit $\kboltz$ als konstantem Umrechnungsfaktor verknüpft ist, dann scheint das obige Transformationsgesetz falsch zu sein. In diesem Sinne machte Heinrich Ott 1963 (ebenfalls posthum) einen alternativen Vorschlag, der mit dem Verhalten von Energie unter Lorentz-Transformationen übereinstimmt. Er behauptete nämlich, dass bei der Ableitung des Lorentz-Transformationsgesetzes für Wärme und Temperatur ein grundlegender Fehler unterlaufen sei. Seiner Meinung nach sollte sich die Temperatur stattdessen wie folgt transformieren:

\begin{equation*}\label{reltemphigher}
  T = \frac{T_0}{\sqrt{1-\frac{v^2}{c^2}}}\,,
\end{equation*}
%
um eine Übereinstimmung mit dem zweiten Hauptsatz der Thermodynamik zu erreichen. Dieser widersprüchliche und alternative Vorschlag wird immer noch in einigen neueren Veröffentlichungen vertreten.

Um die Verwirrung zu vervollständigen, schlug Peter T.\ Landsberg 1966 eine neue Theorie vor, bei der die Temperatur eine Lorentz-Invariante ist:

\begin{equation*}\label{reltempequal}
  T = T_0\,.
\end{equation*}
%
Es besteht also kein Konsens über die Temperatur eines sich bewegenden Körpers. Dies scheint auf unterschiedliche Definitionen für Temperatur, Thermometer, Wärmeübertragung und Arbeit zurückzuführen zu sein. Aber vielleicht ist genau das die Lösung: Alle oben genannten Transformationsgesetze könnten anwendbar sein, abhängig von den ursprünglichen Annahmen~-- eine Schlussfolgerung, zu der Einstein 1952 gegen Ende seines Lebens gelangt zu sein scheint.


\subsection*{Relativistisches ideales Gas}

Es sollte darauf hingewiesen werden, dass Elektron-Positron-Paar-Plasmen, die unter experimentellen Bedingungen erzeugt wurden, Geschwindigkeitsverteilungen gemäß relativistischen Versionen der berühmten Maxwell-Boltzmann-Verteilung aufweisen. Diese wurden erstmals 1911 von Ferencz Jüttner ermittelt und werden als Maxwell-Jüttner-Verteilungen bezeichnet.

Zusammenfassend lässt sich sagen, wie im gerade betrachteten Beispiel, dass einige der Hexadekachor-Tetraeder noch keine sauber definierten vereinheitlichten Theorien bilden. Interessanterweise weisen sie damit vermutlich auf einige ungelöste Rätsel hin, die möglicherweise mit dem Fehlen einer endgültigen physikalischen Theorie, der schwer fassbaren \textit{theory of really everything} (TORE), also einer Theorie von wirklich Allem, zusammenhängen.
