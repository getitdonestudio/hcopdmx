
\newpage \addsec{TORE: Theorie von wirklich Allem $(G,c^{-1},h,\kboltz)$}
\label{sec:1111}

Das äußerste Tetraeder des Hexadekachors, das alle anderen 15 Tetraeder umfasst, sollte die ultimative \emph{theory of really everything} (TORE) sein, also die \emph{Theorie von wirklich Allem}. Diese wäre die vereinheitlichte Theorie hinter allen vier Planckschen Naturkonstanten $(G,c^{-1},h,\kboltz)$, die dieser 1899 nutzte, um universelle Einheiten für Länge, Zeit, Masse und Temperatur zu definieren. Obwohl uns eine experimentell verifizierbare TORE eindeutig fehlt und obwohl es anderseits nicht einmal klar ist, dass sie existieren muss, gibt es mittlerweile eine ganze Reihe wichtiger Hinweise. Sie stehen alle im Zusammenhang mit Schwarzen Löchern, die eine fundierte Vorhersage von Einsteins allgemeiner Relativitätstheorie $(G, c^{–1}, 0, 0)$ sind. Deren tatsächliche physische Existenz wurde in den vergangenen Jahren zweifelsfrei experimentell nachgewiesen.


\subsection*{Schwarze Löcher}

Karl Schwarzschild sagte 1916 Schwarze Löcher vorher, nachdem er Einsteins Feldgleichungen der allgemeinen Relativitätstheorie $(G,c^{-1},0,0)$ in einem Schützengraben einer eingehenden Analyse unterzogen hatte. Nach ihm ist ein Schwarzes Loch ein Raum-Zeit-Gebiet mit einem so starken Gravitationsfeld, dass alles einschließlich des Lichts daran gehindert wird, daraus zu entkommen.

Stephen Hawking entdeckte jedoch 1974, dass die Einbeziehung der Quantenfeldtheorie $(0,c^{-1},h,0)$ zu etwas anderen Vorhersagen führt. Hawking konnte theoretisch beweisen, dass Schwarze Löcher eine absolute Temperatur $T_\mathrm{H}$ haben, die heute als \emph{Hawking-Temperatur} bezeichnet wird, und folglich Wärmestrahlung emittieren. Dementsprechend sind sie nicht ganz so \enquote*{schwarz}, wie sie zunächst scheinen, und ihre korrekte theoretische Beschreibung erfordert die Thermodynamik $(0,0,0,\kboltz)$!


\subsection*{Hawking-Strahlung}

Die Vakuumpolarisation führt zu einem \enquote*{glühenden Horizont}. Mit dieser Teilchenstrahlung aus dem Horizont heraus ist eine Temperatur des Schwarzen Lochs verbunden, die Hawking-Temperatur $T_{\text{H}}$: Schwarze Löcher kann man im Prinzip \emph{sehen}. Die Existenz der Hawking-Strahlung wurde jedoch noch nicht experimentell bestätigt. Die Einführung einer Temperatur bedeutet, dass es in der Nähe eines Schwarzen Lochs ein Vielteilchensystem gibt.


\subsection*{Hawking-Temperatur}

Die berühmte Formel von Hawking für die Temperatur $T_{\text{H}}$ eines Schwarzen Lochs als Funktion seiner Masse $M$ verknüpft nun alle vier fundamentalen Konstanten von Planck:

\begin{equation*}\label{eq:HawkingT}
  T_{\text{H}} = \frac{1}{8 \pi}\, \frac{\hslash\, c^3}{\kboltz\, G}\, \frac{1}{M}
\end{equation*}
%
Wenn $\hslash=\frac{h}{2\pi}\rightarrow0$, dann $T_\text{H}\rightarrow0$, und somit gäbe es keine Hawking-Strahlung: Es handelt sich also wirklich um einen Quanteneffekt!

Aufgrund der Vakuumpolarisation ist die Quantenfeldtheorie von Natur aus eine Vielteilchentheorie. In Horizontnähe eines Schwarzen Lochs können virtuelle Teilchen real werden. Die entscheidende Einsicht ist daher: Die Zusammenführung von Quantenfeldtheorie und Gravitation zwingt uns, auch die Temperatur und die Thermodynamik zu berücksichtigen.


\subsection*{Bekenstein-Hawking-Entropie}

Bekenstein postulierte: Da Schwarze Löcher aufgrund der Hawking-Strahlung intrinsisch thermodynamische Objekte sind, müssen sie eine Entropie besitzen. Und diese nimmt gemäß dem zweiten Hauptsatz der Thermodynamik immer zu.

Daraus ergibt sich ein zweites bahnbrechendes Ergebnis, das alle vier Fundamentalkonstanten miteinander verknüpft: die Entropie $S_{\text{BH}}$ eines Schwarzen Lochs nach Bekenstein und Hawking. Sie besagt, dass $S_{\text{BH}}$ proportional zur Fläche $A$ des zweidimensionalen Horizonts des Schwarzen Lochs ist:

\begin{equation*}
  S_{\text{BH}} = \frac{\kboltz\, c^3}{\hslash\, G}\, \frac{A}{4}
\end{equation*}
%
Dies ist seltsam und mysteriös und seit seiner Entdeckung Gegenstand hitziger Debatten: In \enquote*{üblichen} thermodynamischen Systemen ist die Entropie immer extensiv, d.\,h.\ proportional zu einem dreidimensionalen Volumen. Ist dies der schlagende Beweis, der auf die Existenz der TORE, also der endgültigen \emph{Theorie von wirklich Allem}, hinweist?
