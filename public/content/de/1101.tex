
\newpage \addsec{Allgemeine Relativitätstheorie und statistische Mechanik $(G,c^{-1},0,\kboltz)$}
\label{sec:1101}

Bereits die Vereinigung von spezieller Relativitätstheorie und statistischer Mechanik, repräsentiert durch das Tetraeder $(0,c^{-1},0,\kboltz)$, führt zu Problemen wie der korrekten Abhängigkeit der Temperatur vom Bezugssystem, die bis heute nicht auf allgemein akzeptierte Weise gelöst wurden. Wie zu erwarten, verbessert sich die Situation nicht, wenn auch die Schwerkraft einbezogen wird: Ein überzeugender Rahmen für die Vereinigung von Einsteins allgemeiner Relativitätstheorie und Boltzmanns statistischer Mechanik, der dem Tetraeder $(G,c^{-1},0,\kboltz)$ entspricht, muss erst noch entwickelt werden. Tatsächlich wurde dieses Problem schon früh von Max Planck und Albert Einstein erkannt, und es wurden einige Anstrengungen unternommen, um es zu lösen~-- jedoch ohne, dass man sich bisher vollständig einigen konnte.


\subsection*{Kosmologie und Temperatur}

Viele Probleme in der Kosmologie erfordern die Verwendung von Werkzeugen und Ideen aus sowohl der allgemeinen Relativitätstheorie als auch der Thermodynamik. Tatsächlich wurde der Urknall als sehr naheliegende Vorhersage der klassischen Relativitätstheorie entdeckt. Das physikalische Bild hinter der \emph{anfänglichen} Raum-Zeit-Singularität ist das eines extrem heißen, dichten Plasmas aus Strahlung und Elementarteilchen, das sich in und mit der Raumzeit als solcher extrem schnell ausdehnt. Die Entwicklung zum heutigen Zustand des Universums, etwa \num{13.8} Milliarden Jahre nach dem Urknall, wird für gewöhnlich als eine fortwährende Abkühlung aller Materie und Strahlung verstanden. Dementsprechend wird in der chronologischen Beschreibung der Entwicklung des Universums die Temperatur oftmals als Parameter verwendet. Für den Zeitpunkt des Urknalls wird angenommen, dass dort die Strahlungstemperatur bei etwa \qty{e32}{\kelvin} lag. Nach etwa \num{380000} Jahren hatte sich das Plasma auf \qty{3000}{\kelvin} abgekühlt, und Photonen begannen, sich frei durch den Raum zu bewegen.

Diese Strahlung ist heute noch als kosmische Mikrowellenhintergrundstrahlung (CMB) beobachtbar, aber ihre Temperatur beträgt jetzt nur noch \qty{2.7}{\kelvin}. Trotzdem ist nicht eindeutig klar, ob dieses kosmologische Bild wirklich in das Tetraeder $(G,c^{-1},0,\kboltz)$ aufgenommen werden sollte, im Gegensatz zu $(G,c^{-1},h,\kboltz)$, da die Strahlung letztlich auf Quanteneffekte zurückzuführen ist. Wir haben dies jedoch getan, weil die Kosmologie eine greifbare Beziehung zwischen Temperatur und allgemeiner Relativitätstheorie herstellt.
