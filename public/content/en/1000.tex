
\newpage \addsec{Newtonian gravity $(G,0,0,0)$}
\label{sec:1000}

One of Newton's greatest achievements was the discovery of his general law of gravitation. In modern terms, it states that every point mass in the universe \emph{instantaneously} attracts every other point mass with a force that is proportional to the product of their masses $m$ and $M$, and inversely proportional to the square of the distance $r$ between them:

\begin{equation*}\label{eq:gravitation}
  F=G\,\frac{Mm}{r^2}\,.
\end{equation*}
%
The associated constant of proportionality is the gravitational constant $G$, much later called Newton's constant in his honour. Experimentally, $G$ was first, unintentionally and indirectly, determined as late as 1798 by Henry Cavendish, when he performed his famous experiment to measure the density of the Earth. Its current value is

\begin{equation*}
  G = \SI{6.67430(15)d-11}{\cubic\metre\per\kilogram\per\square\second}\,,
\end{equation*}
%
with a surprisingly large relative uncertainty of \num{2.2d-5}. As such, it is the only one of the hexadecachoron's four fundamental constants $G$, $c^{-1}$, $h$ and $\kboltz$ that has \textit{not} yet been fixed to an exact value, and where various experimental efforts are under way to improve its accuracy.

In the hexadecachoron model, Newtonian gravity corresponds to the spiky tetrahedron $(G,0,0,0)$  pointing down towards the floor. In combination with Newtonian classical mechanics, i.\,e.\ the innermost tetrahedron $(0,0,0,0)$, this theory can be used to describe with astonishing accuracy both the falling of objects on Earth and the motion of celestial objects such as moons, planets, meteorites and satellites in the planetary system.


\subsection*{Newton's theory of gravitation}

In the third book \enquote{De mundi systemate} of his \enquote{Philosophiæ Naturalis Principia Mathematica}, Isaac Newton deduces the motion of the planets and their satellites from his law of gravitation. According to Newton, the resulting force keeps all celestial objects in their orbits. It is inversely proportional to the square of the distance between the centres of mass of two objects with a spherically symmetric mass distribution:

\begin{quote}
  If the matter of two spheres that are mutually heavy towards each other is homogeneous in the areas that are at equal distance from the centre, the weight of each sphere will behave inversely proportional to the square of the distance between their centres. (Newton 1687)
\end{quote}


\subsection*{Equivalence principle}

An integral part of Newton's theory of gravitation is the weak equivalence principle, which states that the inertial and gravitational masses of every object are identical. Galileo had already demonstrated experimentally that two objects with \emph{unequal} masses fall at the same speed, in principle. The necessary proof of this equivalence was provided experimentally by Newton using a special pair of pendulums. To this day, this equivalence is still being investigated using highly complex experiments. Yet a difference between heavy and inertial mass has not been found. A particularly impressive demonstration was shown during the lunar landing mission Apollo 15: a feather and a hammer, dropped simultaneously from the same height, hit ground at the same time.


\subsection*{Action at a distance and Einsteinian gravity}

For Newton, the law of gravitation had an exclusively relational character, he did not introduce the gravitational constant $G$. However, he did derive epochal consequences from the proportionality of the universal gravitation. First and foremost, every planet moves in an elliptical orbit around the sun, which is located at one of the ellipse's focal points. Newton thus formulated a dynamic and mathematical derivation for the already empirically known orbit of planets: \enquote{whereas Kepler guessed right at the Ellipsis} (Newton to Halley 1686). Despite the revolutionary successes, the character of the instantaneous long-distance effect in his law of gravitation remained extremely mysterious and questionable for many of Newton's contemporaries and successors, up to and including Einstein. The latter resolved the mystery after developing his general theory of relativity, see $(G,c^{-1},0,0)$, which is also known as Einstein's theory of gravitation.
