
\newpage \addsec{Special relativity and statistical mechanics $(0,c^{-1},0,\kboltz)$}
\label{sec:0101}

How does temperature change under Lorentz transformations? Or, more bluntly: What is the temperature of a moving body? A generalised theory of thermodynamics properly unified with the principles of special relativity, i.\,e.\ a theory of relativistic thermodynamics, remains a subject of hot debate since its initial proposal by Max Planck and his doctorate student Kurd von Mosengeil, and independently by Albert Einstein in 1907. In the hexadecachoron model, this theory would correspond to the tetrahedron labelled by $(0,c^{-1},0,\kboltz)$.


\subsection*{Temperature confusion}

Which thermodynamic quantities remain invariant under Lorentz transformations? Arguing with the statistical definition of the entropy~-- the number of possible microscopic states of a given macroscopic system~-- all authors agree on its relativistic invariance. The same consensus, however, is not reached for temperature, pressure and their related thermodynamic potentials. According to von Mosengeil’s posthumous thesis (he tragically died at age 22 when hiking in the Alps):

\begin{quote}
  Two bodies, which the resting observer describes as equally hot, [can] appear differently hot to a moving observer [\dots], namely if the bodies have different velocities. The temperature of a body will always appear highest to the observer who is resting relative to him. (Von Mosengeil 1907)
\end{quote}
%
According to him, temperature transforms like the radiation intensity of a moving black body as

\begin{equation*}\label{reltemplower}
  T = T_0\,\sqrt{1-\frac{v^2}{c^2}}\,,
\end{equation*}
%
where $T$ and $T_0$ are the temperatures inside the moving and rest frame, respectively, and $v$ is the relative frame velocity. This transformation law is accepted in much of the recent literature, although not universally so.

And indeed, note the following puzzling aspect: if, as it is often claimed, temperature is simply a form of energy, related through $\kboltz$ as a constant conversion factor, then the above transformation law appears to be wrong. In line with this, an alternative proposal agreeing with the behaviour of energy under Lorentz transformations was made by Heinrich Ott in 1963 (posthumously, once again). He claimed a fundamental mistake in the derivation of the Lorentz transformation law for heat and temperature. According to him temperature should instead transform as

\begin{equation*}\label{reltemphigher}
  T = \frac{T_0}{\sqrt{1-\frac{v^2}{c^2}}}\,,
\end{equation*}
%
in order to reach consistency with the second law of thermodynamics. This contradictory and alternative proposal is also supported some recent publications. Finally, completing the confusion, in 1966 Peter T.\ Landsberg proposed a new theory, where the temperature is a Lorentz invariant:

\begin{equation*}\label{reltempequal}
  T = T_0\,.
\end{equation*}
%
Hence there has been substantial lack of consensus on the temperature of a moving body. It seems to be rooted in differing definitions for temperature, thermometers, heat transfer and work. But perhaps this is precisely the solution: All of the above transformation laws might be applicable, depending on the initial assumptions~-- a conclusion Einstein himself seems to have reached in 1952 near the end of his life.


\subsection*{Relativistic ideal gas}

It should be pointed out that electron–positron pair plasmas, produced in experimental conditions, have been show to follow speed distributions according to relativistic versions of the famous Maxwell-Boltzmann distribution. These have been first obtained by Ferencz Jüttner in 1911, and are called Maxwell-Jüttner distibutions.

In conclusion, we see that some of the hexadecachoron's tetrahedra do not, or at least not yet, correspond to important unified theories. Interestingly, however, they do point to some unresolved puzzles, possibly related to our lack of a final theory of physics, the elusive theory of really everything (TORE).
