
\newpage \addsec{Newtonian mechanics $(0,0,0,0)$}
\label{sec:0000}

Imagine a universe in which gravity can be neglected. Where things move very slowly compared to the speed of light. In which quantum effects are too small to be perceived. And in which the number of particles is very limited. Such a universe would be fully described by the laws of Galilei's space-time model and Newtonian mechanics. In the hexadecachoron model, this corresponds to the theory in which the gravitational constant $G$, the inverse of speed of light $c^{-1}$, Planck's quantum of action $h$ and the Boltzmann constant $\kboltz$ are all set to zero: this is the small tetrahedron at the very centre of the model. It symbolises the origin of modern physics in both a conceptual as well as a historical manner.

All objects in this universe have a certain position in an absolute space. Time is a parameter that uniformly and perpetually counts the passing of seconds, minutes and hours everywhere. Every movement corresponds to a fixed path, which can be calculated. And every relation between the particles is determined. Sounds familiar? As we all know, this theory does not describe reality. It is merely a first, yet controlled approximation to the rich physical world around us.


\subsection*{Newton's foundations}

From the perspective of modern physics, Newtonian mechanics deals with the motion of sufficiently slow classical bodies under the influence of a system of arbitrary physical forces. Thus, pure Newtonian mechanics per se knows no gravity, no light, no quantum of action, and does not care whether there are just a few or a very large number of particles. It is the theory $(0,0,0,0)$ of the hexadecachoron model and marks the origin of a Cartesian coordinate system with the four fundamental natural constants defining its axes. At the same time, it assigns all other theories their proper place in the model.

Newton summarized his findings in three laws. Arguably, the second law is the most important, and states that the force $\vec{F}$ exerted on a body equals that body's mass $m$ times its acceleration $\vec{a}$:

\begin{equation*}
  \vec{F} =m\, \vec{a}\,.
\end{equation*}
%
Clearly, no fundamental constants appear.

If gravity is added, Newtonian mechanics remains intact. It is merely extended to the theory $(G,0,0,0)$, which adds Newton's law of gravity and the conceptual idea of the identity between inertial and heavy mass. Hence, this theory defines a new tetrahedron of the model.


\subsection*{Galilei's space-time model}

Newtonian mechanics is based on Galilei's model of space-time. It assumes that two inertial systems are physically indistinguishable. Take Galilei's ship moving uniformly and rectilinearly in his famous thought experiment: Standing below decks, all hatches closed and no smartphone at hand, one can determine one's own movement relative to the ship, but not relative to the sea. His thought experiment was so important to Galilei, that he placed it at the beginning of his \enquote{Dialogue concerning the two chief world systems}, between the figures of Aristotle, Ptolemy and Copernicus. Moreover, Newtonian mechanics asserts an absolute time, identical in all inertial systems. Velocities of inertial systems add up linearly. This is sufficiently accurate for the description of everyday mechanical systems.


\subsection*{Subsequent corrections}

However, Galilei's space-time model leads to logical contradictions with Maxwell's equations of electrodynamics. Thus, fundamental corrections for high velocities are needed. These can only be understood within the framework of Einstein's special relativity, i.\,e.\ the theory $(0,c^{−1},0,0)$.

Classical mechanics experiences even more drastic corrections on atomic and subatomic length scales. It loses its deterministic character and was reformulated as a theory of probability amplitudes within the framework of quantum mechanics $(0,0,h,0)$ in the first third of the 20th century.

Even before that, Maxwell and Boltzmann and many others showed that thermodynamics may be understood as a statistical theory of very many classical, Newtonian particles: Classical statistical mechanics: $(0,0,0,\kboltz)$.

Hence, all these corrections to classical mechanics point to new tetrahedra of the model. The two theories $(0,0,0,0)$, Newtonian mechanics, and $(G,0,0,0)$, Newtonian gravitational theory, were developed by Newton at the same time; so here, the systematics of the hexadecachoron contradicts the history of physics in a certain way. Newton's theory is thus visualized as two tetrahedra of the model, but he never made this distinction himself. The corrections $(0,c^{−1},0,0)$, $(0,0,h,0)$ and $(0,0,0,\kboltz)$ on the other hand, as is well known, were only obtained around the turn of the 20th century; here, the model reflects the history of knowledge.
