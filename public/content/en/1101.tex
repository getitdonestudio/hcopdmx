
\newpage \addsec{General relativity and statistical mechanics $(G,c^{-1},0,\kboltz)$}
\label{sec:1101}

Already the union of special relativity and statistical mechanics, represented by the tetrahedron $(0,c^{-1},0,\kboltz)$, leads to puzzles such as the correct dependence of temperature on the frame of reference, that have not been resolved to-date in a generally accepted way. As one might have expected, the situation does not improve when including gravity: A convincing framework for uniting Einstein's general theory of relativity and Boltzmann's general theory of statistical mechanics, corresponding to the tetrahedron $(G,c^{-1},0,\kboltz)$, has yet to be developed. In fact, this problem was certainly spotted by Max Planck and Albert Einstein very early on, and quite a few effort went into it, with no generally accepted solution.


\subsection*{Cosmology and temperature}

Of course, a large number of problems in cosmology require using tools and ideas from both general relativity and thermodynamics. In fact, the Big Bang was discovered as a natural prediction from classical relativity. The physical picture behind the \emph{initial} space-time singularity is that of an extremely hot, dense plasma of radiation and elementary particles, rapidly expanding in and with space-time itself. The evolution to the present state of the universe, some \num{13.8} billion years after the Big Bang, is often understood as an ongoing \emph{cooling down} of all matter and radiation. Accordingly, in the chronological description of the universe's evolution, one often uses temperature as a parameter. At the time of the Big Bang, the radiation temperature is assumed to have been around \qty{e32}{\kelvin}. After roughly \num{380000} years, the plasma had cooled down to \qty{3000}{\kelvin}, and photons started to travel freely through space.

Today this radiation is still observable as the cosmic microwave background (CMB), but now its temperature is a mere \qty{2.7}{\kelvin}. However, it is not clear whether this cosmological picture should really be included at the $(G,c^{-1},0,\kboltz)$ tetrahedron as opposed to $(G,c^{-1},h,\kboltz)$, since radiation is ultimately due to quantum effects. We did so, because cosmology provides a tangible relation between temperature and general relativity.


\subsection*{Astrophysics and temperature}

Astrophysics is another field within physics where both general relativity and the temperature concept play crucial roles. For example, young neutron stars have surface temperature of around \qty{e7}{\kelvin}, but subsequently cool down continuously as no new heat is generated in their interior. However, as in the case of cosmology, it is not completely clear whether astrophysics should not rather be assigned to the $(G,c^{-1},h,\kboltz)$ tetrahedron, since results from quantum physics must often be used as well.

