

\newpage \addsec{Theory of really everything $(G,c^{-1},h,\kboltz)$}
\label{sec:1111}

The outermost tetrahedron of the hexadecachoron, encompassing all other 15 tetrahedra, should be the ultimate \emph{theory of really everything}: TORE. It should be the unified theory behind all four of Planck's natural constants $(G,c^{-1},h,\kboltz)$ that he singled out in 1899 in his quest to define universal units for length, time, mass and temperature. While we are certainly still lacking an experimentally verifiable TORE, and while it is of course not even clear that it must exist, certainly a number of important hints have been discovered. They are all related to \emph{black holes}, which are a sound prediction of Einstein's general theory of relativity $(G,c^{-1},0,0)$. Their physical existence has been experimentally established beyond reasonable doubt in recent years.


\subsection*{Black holes}

Karl Schwarzschild predicted black holes in 1916 after he had subjected Einstein's field equations of general relativity $(G,c^{-1},0,0)$ to an in-depth analysis in a  World War I trench. According to him, a black hole is a space-time region with a gravitational field so intense that everything, including light, is prevented from escaping from it.

However, in 1974 Stephen Hawking discovered that the inclusion of quantum field theory $(0,c^{-1},h,0)$ leads to dramatically different predictions. Hawking was able to theoretically prove that black holes have an absolute temperature $T_\mathrm{H}$, now called \emph{Hawking temperature}, and consequently emit thermal radiation. Accordingly, they are not as \enquote*{black} as they seemed at first, and their correct theoretical description requires thermodynamics $(0,0,0,\kboltz)$!


\subsection*{Hawking radiation}

Vacuum polarization leads to a \emph{glowing horizon}. Associated to this radiation of particles out of the horizon is a \emph{temperature}, called Hawking temperature $T_{\text{H}}$, of the black hole: One can \enquote*{see} black holes in principle. Hawking radiation has not yet been confirmed experimentally. Introducing a temperature means that there is a many-body system close to black holes. 


\subsection*{Hawking temperature}

Hawking's famous formula for the temperature $T_{\text{H}}$ of a black hole as a function of its mass $M$ links all four of Planck's fundamental constants:

\begin{equation*}\label{eq:HawkingT}
  T_{\text{H}} = \frac{1}{8 \pi}\, \frac{\hslash\, c^3}{\kboltz\, G}\, \frac{1}{M}
\end{equation*}
%
If $\hslash=\frac{h}{2\pi}\rightarrow0$, then $T_\text{H}\rightarrow0$, and then there is no Hawking radiation: it's really a quantum effect!

Because of vacuum polarization, quantum field theory is by construction a many-particle theory. Near the horizon of a black hole, virtual particles may become real. Thus, the crucial take-home message is~-- quantum field theory with gravity forces us to consider temperature and thermodynamics.


\subsection*{Bekenstein-Hawking entropy}

Bekenstein postulated: Since black holes are intrinsically thermodynamic objects due to Hawking radiation, they must have entropy. And the latter always increases in line with the 2nd law of thermodynamics.

This allows for a second seminal result that links all four fundamental constants: the entropy $S_{\text{BH}}$ of a black hole due to Bekenstein and Hawking. It states that $S_{\text{BH}}$ is proportional to the area $A$ of the black hole's two-dimensional horizon:

\begin{equation*}
  S_{\text{BH}} = \frac{\kboltz\, c^3}{\hslash\, G}\, \frac{A}{4}
\end{equation*}
%
This is strange and mysterious and has been a subject of heated debate ever since its discovery: In \enquote*{usual} thermodynamic systems the entropy is always extensive, i.\,e.\ proportional to a three-dimensional volume. Is this the smoking gun that points to TORE, the final \emph{theory of really everything}? 
