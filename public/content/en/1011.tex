
\newpage \addsec{Non-relativistic quantum gravity and temperature $(G,0,h,\kboltz)$}
\label{sec:1011}

We are currently lacking a generally accepted theory of quantum gravity, even though there are experimentally unverified and theoretically incomplete candidates such as string theory or loop quantum gravity. This situation does not improve,  of course, if one includes temperature, even though there are interesting conjectures about the consequences of such an all-encompassing \emph{theory of really everything} (TORE), involving all four constants $(G,c^{-1},h,\kboltz)$, such as Hawking radiation or Bekenstein-Hawking entropy.

One might hope that a non-relativistic (i.\,e.\ $c^{-1}=0$) version of the sought TORE model, based on the constants $(G,0,h,\kboltz)$, should be easier to find. However, this does not seem to be the case, and basically no viable suggestions have been made. There are nevertheless very interesting, topical experiments relating quantum particles, temperature and gravity: These are studies of ultracold neutrons (UCN), going back to early ideas of Enrico Fermi in the mid-thirties.

UCN are essentially free and extremely slow (= cold) neutrons trapped in suitable containers: Their walls perfectly reflect any incident neutrons. This allows to treat UCN as a kind of dilute ideal gas with a temperature of less than \qty{4}{\milli\kelvin} (Millikelvin). Since the kinetic energy of the UCN is very small, the gravitational force plays a significant role in the description of the system. These experiments go by the name of \emph{neutron bottle experiments.}

There is an interesting puzzle related to the lifetime of free neutrons when comparing the values measured from the bottle experiments to the one measured in neutron beam experiments. The lifetime currently (2024) reported by the \emph{particle data group} is \qty{878.4(0.5)}{\second} (roughly 15 minutes). This average value, which takes into account only measurements from the bottle experiments, is about \qty{9}{\second} off the typical values obtained from the beam experiments. There is an ongoing debate on this on how to get a better experimental but especially theoretical understanding of this discrepancy.
