
\newpage \addsec{Theory of everything $(G,c^{-1},h,0)$}
\label{sec:1110}

Considered by many the so far undiscovered holy grail of modern physics. This theory would include the three fundamental constants $G,c^{−1}$ and $h$, omitting Planck's fourth choice $\kboltz$, the Boltzmann constant. There are quite a few ambitious candidates for the throne, most notably string theory and loop quantum gravity. However, it has not yet been possible to test whether either of these theories is correct~-- in the end probably both are wrong, or at least incomplete. By combining the three fundamental natural constants $G$, $c^{-1}$ and $h$ we can derive Planck's natural units for space, time and mass, disregarding temperature. It turns out that quantum gravity becomes relevant at orders of magnitude that correspond to these Planck units.

Unfortunately, these scales are extremely far away from anything that we can directly deduce experimentally. For example, the Planck length is about $10^{−35}$ metres, which is an almost unthinkable scale $10^{20}$ times smaller than the size of the proton.


\subsection*{A world formula?}

What exactly is the theory $(G,c^{-1},h,0)$ of the hexadecachoron of physics? It should be mathematically consistent and experimentally predictive, and thus falsifiable. In German, this sought-after theory is often abbreviated \emph{Weltformel} (\emph{world formula}), although it is a priori unclear whether it is based on a single formula, even if this theory exists. In English, on the other hand, it is brazenly called \emph{theory of everything} (TOE).

The theory of everything seeks the unification of quantum field theory and general relativity. Gravity should be \emph{quantised} and the constant $h$ should somehow be related to gravity. Likewise $G$ should be connected to quantum field theory. However, if general relativity is \emph{quantised} in a naive way in analogy to electrodynamics, e.\,g.\ by summing up all possible metrics, a catastrophe occurs: An infinite number of uncontrollable \emph{divergences} appears, entirely destroying the predictive power of the theory.

The two most prominent of all currently pursued approaches that attempt to circumvent this difficulty are superstring theory and loop quantum gravity. To date, both share the \enquote*{feature} of not being connected to experiment. Related to this, they each come in a large variety of different versions. They are thus more like classes of theories than specific theories.


\subsection*{Superstring theory}

Superstring theory has its roots in quantum field theory, whose point particles are replaced by vibrating strings. The different particles of nature are then represented by the allowed oscillatory modes of these strings. Through this approach, the said divergences disappear, and particle and gravitational physics are united. A hypothetical \emph{graviton} appears, all worlds become ten or even eleven-dimensional, and our specific world is apparently only one of at least $10^{500}$ many. Some of these worlds are at least similar to the standard model of particle physics, but unfortunately an exact match has not been discovered yet. A speculative way out is the idea of the multiverse, where many of these possible worlds exist in parallel universes that, somewhat depressingly, do not communicate with each other. In this scenario, the many yet unexplained parameters of the standard model would ultimately be randomly set or else fixed in that they enable intelligent life (anthropic principle). Not so nice.


\subsection*{Loop quantum gravity}

Loop quantum gravity has its roots in general relativity. The latter is directly quantised; this leads to a kind of \emph{quantum foam} on the Planck scale in the range of $10^{-35}$ metres. There are many partially contradictory approaches, and none of them has yet achieved a natural unification with the quantum field theories of the standard model. These have to basically be \emph{added by hand}. Not nice, either.


\subsection*{Open questions \dots}

Is there any need at all for a theory that combines quantum field theory and general relativity? There is a broad and affirmative consensus in physics on this question. For instance, given the current framework of physical theory, it is obvious that both the detailed description of black holes and that of the big bang require quantum mechanics. In addition, it is at least experimentally clear that something fundamental is missing from the current body of theories in physics: About 96 percent of the matter and energy of our universe seem to be unknown, or, as one says, \enquote*{dark}.
