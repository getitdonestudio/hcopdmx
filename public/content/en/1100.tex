
\newpage \addsec{General relativity $(G,c^{-1},0,0)$}
\label{sec:1100}

In order to unite Newtonian gravity with special relativity, both natural constants $G$ and $c^{−1}$ must be taken into account. This is exactly what Einstein's general theory of relativity (1915) does. It transforms the static four-dimensional Minkowskian space-time into a dynamic and curved object, whose shape is uniquely fixed by Einstein's field equations.

In curved space-time, objects always move on the shortest possible path as determined by an indefinite metric. A historically important example is Mercury's orbit around the Sun. According to Einstein, the Sun curves space-time in a way that results in a nearly elliptic orbit, up to a small perihelion shift. Contrary to Newton, the reason for Mercury's curved path is not some mysterious gravitational force emanating from the Sun, but simple economy. An analogy is the trajectory of an aeroplane from Prague to Seoul: It is curved, but by no means due to a hypothetical force at the North Pole.


\subsection*{Action principle}

Why was Einstein so sure that he had to develop a new theory of gravity that would self-consistently unite the constants $c^{−1}$ and $G$? After hiring him to Berlin, Planck strongly recommended him to not waste his time, but to rather focus on the development of quantum mechanics. However, Einstein was acutely aware that Newtonian gravitational theory and special relativity are fundamentally incompatible. For example, in the former, a change in the distance between two masses results in an instantaneously experienced change in the gravitational force acting between them. In contradistinction, in special relativity, signal transmission at velocities greater than $c$ is impossible.

This fundamental feature, that Newton's law of gravity depends merely on location but not on time, had to be eliminated by general relativity. In hindsight, the most elegant way to derive the latter theory involves a suitable action principle: One considers a body's movement over a period of time and determines the actual path travelled according to a suitable criterion: \enquote{the one by which the quantity of action is the least!} (Maupertuis 1744)


\subsection*{Einstein's field equations}

This action principle then implies the famous Euler-Lagrange equations:

\begin{equation*}\label{eq:euler-lagrange}
  \frac{d}{dt}\frac{\partial L}{\partial \dot{q}_i}-\frac{\partial L}{\partial q_i} = 0\,.
\end{equation*}
%
Here, $q_i(t)$ is the coordinate of the $i$-th particle at time $t$, and $L$ the Lagrangian, given by the difference between the particles' kinetic and potential energies. The Euler-Lagrange equations precisely reproduce Newton's equations of motion. Incidentally, note that this derivation is based on an erroneous fundamental asymmetry of space and time: the different states of a particle are only integrated over time. Generalizing this idea to Einsteinian gravity, while repairing the space-time asymmetry, one is looking for a suitable action $S$ that allows for its minimisation under arbitrary variations of the metric:

\begin{equation*}\label{eq:einstein-hilbert-action}
  \frac{\delta S}{\delta g_{\mu\nu}} = 0\,.
\end{equation*}
%
The actual metric is then once again calculated by using an analogue of the Euler-Lagrange equations. The correct choice is now called the Einstein-Hilbert action and leads to Einstein's field equations:

\begin{equation*}\label{eq:field-equations}
  R_{\mu\nu}-\frac{1}{2}g_{\mu\nu}R = \frac{8\pi G}{c^4} T_{\mu\nu}\,.
\end{equation*}
%
They describe in a mathematically rigorous fashion that matter ($T_{\mu\nu}$) bends ($R_{\mu\nu}$ and $R$) space-time ($g_{\mu\nu}$). Objects move in this metric on curved tracks without experiencing any force: gravity simply corresponds to the curvature of space-time.

All objects in the cosmos then simply travel on the shortest and fastest possible connection in a curved space-time. On their way they bend with their mass and energy the space-time structure in their immediate vicinity. Precisely stated: Everything moves on a curved four-dimensional Lorentzian manifold, whose local structure is, in turn, modified by the mover. General relativity thus solves the contradictions between Newton's theory of gravity and special relativity connecting the constants $G$ and $c^{-1}$. This yields the theory $(G,c^{-1},0,0)$ of the hexadecachoron.
