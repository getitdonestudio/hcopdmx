
\newpage \addsec{Quantum statistical mechanics $(0,0,h,\kboltz)$}
\label{sec:0011}

Pairing non-relativistic quantum mechanics $(0,0,h,0)$ and statistical mechanics $(0,0,0,\kboltz)$ led to a perfect marriage: quantum Boltzmann statistics. This even resulted in a wonderful offspring: Boltzmann statistics had to be adapted accordingly as soon as the statistics concerned \emph{identical} particles of either bosonic or fermionic nature.


\subsection*{Quantum Boltzmann statistics}

On a formal level, the partition functions of classical statistical mechanics are easily adapted to quantum mechanics by introducing a density operator $\hat\rho$ for a canonical ensemble of particles as follows:

\begin{equation*}
  \hat \rho=\frac{1}{Z} \exp\left(-\frac{\hat H}{\kboltz T}\right)
    \qquad \textrm{with} \qquad
    Z=\Tr \exp\left(-\frac{\hat H}{\kboltz T}\right),
\end{equation*}
%
where $\hat H$ is the Hamilton operator of the system and $Z$ its partition function, given as a trace over the Hilbert space. Clearly one may interpret $\hat\rho$ as a normalized, operatorial version of the Boltzmann factor, with the explicit appearance of $\kboltz$.

The density operator is obviously normalized to one: $\Tr \hat \rho=1$. In a basis of energy eigenstates, where $\hat H |\psi_n\rangle=E_n  |\psi_n\rangle$, one has

\begin{equation*}
  \hat \rho=\sum_n p_n |\psi_n\rangle \langle \psi_n|
    \qquad \textrm{with} \qquad
    p_n=\frac{1}{Z} \exp\left(-\frac{E_n}{\kboltz T}\right).
\end{equation*}
%
Here $p_n$ is the probability (\emph{not} the probability amplitude!) that the system's state has the energy $E_n$. In a quantum mechanical system, the presence of Planck's constant $h$ is implicit; it will appear in the expression for $\hat H$. However, it does appear explicitly in the \emph{von Neumann equation} for the time evolution of the density operator

\begin{equation*}\label{eq:vNeumann}
  i \hbar \frac{\partial \hat \rho}{\partial t}=\left[\hat H, \hat \rho\right],
\end{equation*}
%
where the brackets stand for a commutator. It may be considered as a generalization of the Schrödinger equation for pure states to an equation of motion for statistically mixed states. Note that in this interpretation $\hat \rho$ encodes a quantum statistical \emph{state} as opposed to a quantum mechanical \emph{operator}, for which a minus sign would appear on the r.\,h.\,s.\ of this evolution operator. Given the density operator $\hat \rho$, one obtains the \emph{von Neumann entropy} $S$ through another trace over the Hilbert space:

\begin{equation*}
  S=-\kboltz \Tr \left( \hat \rho \log \hat \rho \right).
\end{equation*}
%
Note again the explicit appearance of $\kboltz$ in this equation.


\subsection*{Bose-Einstein statistics for bosons}

For the transition from classical to quantum physics, kinetic gas theory played an important role: Max Planck and Albert Einstein used the statistical methods of James Clerk Maxwell and Ludwig Boltzmann to develop a quantum theory of radiation. The quanta of radiation are photons, which are \emph{bosons} of spin one.

To derive it, one considers an ideal quantum gas of bosons coupled to a heat bath and a particle reservoir, and looks at the grand-canonical partition function

\begin{equation*}
  Z_{\mathrm{G}}=\Tr \exp\left(-\frac{\hat H-\mu \bar N}{\kboltz T} \right),
\end{equation*}
%
where the trace is taken over all of the system's energy states as well as over all possible particle numbers $N$, with $\hat N$ being the particle number operator and $\mu$ the chemical potential. Now assuming the basic property that the energy eigenvalues of $\hat H$ can be attained by any number $n_\nu=0,1,2,3, \ldots$ of bosons, the so-called occupation numbers of the $\nu$-th single-particle state of energy $E_\nu$ (such that the total energy eigenvalue of $\hat H$ is $E=\sum_\nu n_\nu E_\nu$ and total particle number operator eigenvalue of $\hat N$ is $N=\sum_\nu n_\nu$), one finds the expectation value

\begin{equation*}
  \langle n_\nu \rangle=
    \frac{1}{\exp\left(\frac{E_\nu-\mu}{\kboltz T} \right)-1}\,.
\end{equation*}
%
This is the Bose-Einstein distribution, which famously first appeared in Max Planck's 1900 law of black body radiation. Planck's constant $h$ is hidden in $E_\nu$.

An exciting application is the Bose–Einstein condensate, predicted in 1924, and first measure in 1995. It constitutes a state of matter in which a gas of bosonic (quasi)particles (= excitations that effectively behave like quantum particles) is cooled down to temperatures below its critical temperature. Under such conditions, a large fraction of the particles \emph{condenses} to the lowest quantum state.


\subsection*{Fermi-Dirac statistics for fermions}

In 1926, Paul Dirac and Enrico Fermi proposed the quantum mechanical version of an ideal gas formed by non-interacting fermions such as electrons, of huge importance in solid state physics. These obey Pauli's exclusion principle, which means that the occupation numbers considered above are now only allowed to be $n_\nu=0,1$. One then finds for the expectation value of the occupation number $n_\nu$ of the $\nu$-th single-particle state of energy $E_\nu$ the Fermi-Dirac distribution

\begin{equation*}
  \langle n_\nu \rangle=
    \frac{1}{\exp\left(\frac{E_\nu-\mu}{\kboltz T} \right)+1}\,.
\end{equation*}

