
\newpage \addsec{Quantum field theory $(0,c^{-1},h,0)$}
\label{sec:0110}

Quantum mechanics was initially formulated in the 1920s in the non-relativistic limit only, i.\,e.\ $c^{−1}=0$, although the correctness of Einstein's special theory of relativity had already been generally accepted for years, with the notable exception of a few antisemitic idiots. In 1928, Dirac succeeded in establishing a curious relativistic wave equation that generalised the Schrödinger equation in a highly non-trivial fashion. It is now engraved in front of Isaac Newton's tomb in Westminster Abbey in the elegant form $i \gamma \cdot \partial\, \psi = m\, \psi$, where $c^{−1}$ and $\hbar=\frac{h}{2 \pi}$ have been set to one. Reinstalling them, Dirac's equation reads

\begin{equation*}
  i\, \hbar\,\, \gamma \cdot \partial\, \psi = m\,c\, \psi\,.
\end{equation*}
%
In particular, it explained the electron's spin and predicted its antiparticle, the positron. A major triumph was Carl Anderson's experimental confirmation of the positron in 1932.

Nevertheless, the interpretation of the Dirac equation as a wave equation led to puzzling contradictions. In addition, problems arose with the quantisation of electrodynamics, which is also a theory where $c^{−1}$ is switched on. As a way out, \textit{quantum field theory} (QFT) was developed in the late 1940's as a relativistically consistent many-particle theory. This eventually led some 20 years later to the standard model of elementary particles, which since then precisely describes all forces of nature with the exception of gravity.


\subsection*{Standard model}

The correct quantisation of electromagnetic fields~-- more precisely, the so-called abelian gauge fields~-- leads to quantum electrodynamics (QED). This is the most thoroughly tested theory ever established. The quanta of these fields are nothing other than Einstein's photons, which also have a spin. But unlike the case of the electron, its value is one. The nuclear forces are very successfully described by quantum field theories as well, employing further spin 1 particles: the eight gluons of the strong nuclear forces and the W$^+$, W$^-$ and Z-bosons of the weak nuclear forces. To describe these particles and the forces of nature corresponding to them, the abelian gauge fields have to be replaced by mathematically much more involved non-abelian fields. Overall, this results in the current standard model of elementary particle physics. Its non-abelian gauge symmetry group is $\mathrm{SU}(3) \times \mathrm{SU}(2) \times \mathrm{U}(1)$, where the first factor is related to the strong nuclear forces (quantum chromo dynamics: QCD), the second to the weak nuclear forces and the third to QED. One open mystery is why nature chooses precisely this gauge group. Another mystery are the unexplained 19 free parameters of the standard model, extended by another seven free parameters for the elusive, slightly massive and chameleon-like neutrinos.

Of particular note is the standard model's Higgs field, which was already predicted in the early 1960s and, after a long search, was finally discovered in 2012 with the Large Hadron Collider (LHC) at CERN, Geneva. It gives most elementary particles a mass, see the $m$ in the Dirac equation above in the case of the spin $\frac12$ particle.


\subsection*{The graviton}

So far it has not been possible to integrate the theory of gravity into the standard model. However, the hypothetical particle carrying the gravitational force has already been christened graviton. If it exists, it should have spin 2. Its inclusion would lead from the tetrahedron $(0, c^{-1},h,0)$ to the tetrahedron $(G,c^{-1},h,0)$ of the hexadecachoron, i.\,e.\ to the theory of everything (TOE).


\subsection*{Rethinking quantum field theory}

As a theory, QFT appears to be somewhat poorly defined, despite the huge number of textbooks written on it. For a mathematician, it is simply ill-defined. The main practical handle to deal with it are elaborate perturbative methods based on Feynman diagrams, or a numerical but non-perturbative reformulation called lattice gauge theory. There are ongoing exciting attempts to develop entirely new approaches to QFT. There are also fascinating recent applications of the formalism of QFT to gravitational wave research in the classical $(G,c^{-1},0,0)$ theory (= general relativity).
