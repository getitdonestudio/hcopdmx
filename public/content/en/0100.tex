
\newpage \addsec{Special relativity and electrodynamics $(0,c^{-1},0,0)$}
\label{sec:0100}

In his \emph{annus mirabilis} 1905, Albert Einstein published four famous papers. In the third one \enquote{On the electrodynamics of moving bodies}, he introduced the special theory of relativity based on two postulates: Inertial frame invariance~-- the laws of physics are invariant in all inertial frames of reference~-- and the constancy of the vacuum speed of light $c$, namely that it is independent of the state of motion of the light source. The idea that $c$ is a fundamental constant for all possible observers, i.\,e.\ independent of their own state of motion, seemed contradictory at least at first sight, and required radical questioning of the assumptions of Galileo and Newton on the structure of space and time. As a consequence of his analysis, Einstein obtained modern physics' perhaps most famous formula

\begin{equation*}\label{emc2}
  E=m\,c^2\,,
\end{equation*}
%
which expresses the equivalence of mass ($m$) and energy ($E$).

The first proof of the finiteness of the speed of light was given by Ole Rømer in 1676, and its first rough estimate was obtained two years later by Huygens. Since then its accuracy was constantly improved, until the speed of light was finally fixed in 1983 to the \emph{exact} value

\begin{equation*}
  c=\SI{299 792 458}{\meter \per \second}\,.
\end{equation*}

In the hexadecachoron model, the relevant parameter actually turns out to be the inverse speed of light $c^{-1}$, since the non-relativistic limit is obtained by taking the speed of light to infinity. Accordingly, special relativity corresponds to the spiky tetrahedron $(0, c^{-1}, 0, 0)$. Of central importance in the hexadecachoron model is that this tetrahedron also includes Maxwell's theory of classical electromagnetism.


\subsection*{Electromagnetism}

Even though the speed of light was anchored as a new fundamental constant in physics with the special theory of relativity, its beginnings date back to the 19th century. In 1873, James Clerk Maxwell not only unified electricity and magnetism, but also identified light as an electromagnetic phenomenon. Since electromagnetism was explained as the physical state of an all-pervading ether, there was an obvious need to experimentally establish its existence. However, the famous interferometer experiments of the 1880s by Edward Morley and Albert Michelson showed that the speed at which light travels does not depend in any way on the motion of the earth through the ether, thereby providing the experimental basis for Einstein's postulates. The complex history of the Lorentz transformations began, which described the exact transformation behaviour of Maxwell's equations.


\subsection*{Lorentz transformations}

The two postulates of special relativity are only compatible if one uses the Lorentz transformations to switch between the coordinates of two inertial frames moving at a constant velocity relative to each other. Galilean transformations then emerge as an approximation for low relative frame velocities. Moreover, the consequences for reference systems moving uniformly relative to each other at velocities comparable to the speed of light are reflected in well-known phenomena such as time dilation or length contraction.


\subsection*{Minkowski space-time}

In 1907, Hermann Minkowski drew the mathematical and conceptual consequences from Einstein's ideas and proposed to replace three-dimensional Euclidean space by four-dimensional space-time. This mathematically precise fusion of space and time brought physics closer to philosophical concepts like the Incan pacha cosmological ideas or the Taoist view of space and time as a \enquote*{complex cosmic web}.

Following Minkowski's formalism, we can generalize the classical three-dimensional momentum $\mathbf{p}$ to a four-vector, whose time component accounts for the energy $E$. In Minkowski space-time, the energy–momentum four-vector is expressed as $P = (E/c, \mathbf{p})$, with squared non-Euclidean norm $P^2=(E/c)^2 - \mathbf{p}^2 = m^2c^2$. For the special case of an object at rest, $\mathbf{p} = 0$, one then indeed finds $E=m c^2$.

Setting, on the other hand, $m=0$, we obtain the relation between the energy and the momentum of massless particles such as photons, the quanta of light and radiation:

\begin{equation*}\label{emc2}
  E=c\, |\mathbf{p}|\,.
\end{equation*}
%
Even though special relativity $(0,c^{-1},0,0)$ seemingly knows nothing about the theory of quantum mechanics $(0,0,h,0)$, which corresponds to an altogether different spiky tetrahedron of the hexadecachoron!

