
\newpage \addsec{Non-relativistic quantum gravity $(G,0,h,0)$}
\label{sec:1010}

Adding the gravitational constant $G$ to non-relativistic quantum mechanics is in principle quite straightforward: One solves the Schrödinger equation for massive particles moving in the Newtonian gravitational potential. A more interesting consideration is to assume the existence of a theory of really everything (TORE), still to be established, and to imagine carefully switching off the inverse speed of light $c^{−1}$ by mathematically taking the limit $c\rightarrow\infty$ in all of that putative theory's formulas.

Does this really not lead to any unexpected measurable effects? Do we really only have the rather boring quantum mechanical scenario outlined at the beginning? The answers are still pending. In this respect, non-relativistic quantum gravity does not currently represent a theory of its own in physics, and there is hardly any research on it. However, it should be mentioned that a number of unconventional approaches to quantum gravity precisely play with the possibility of breaking the Lorentz symmetry between space and time.


\subsection*{Experimental findings}

Since the 1960s, numerous experiments with quantum particles in classical gravitational fields, such as the one of the Earth, have been carried out. Clearly it is essential to use slow quantum particles, singling out neutrons to be particularly suitable. In 1975, for example, Colella, Overhauser and Werner carried out an experiment whose result depended on both Newton's theory of gravity and quantum mechanics. A beam of very slow neutrons was first split and then examined interferometrically. By rotating the measuring device by an angle $\Phi$, they could show that a quantum mechanical phase shift of the slow neutrons occurs due to their interaction with the gravitational field of the Earth. Somewhat disappointingly, the experimental findings are in perfect agreement with the theoretical prediction of the Schrödinger equation equipped with a Newtonian gravitational potential. More recent experiments have even been able to demonstrate the quantisation of weak binding states of ultracold, i.\,e.\ very slow neutrons to the Earth's gravitational field.


\subsection*{More clues}

These experiments are potentially relevant for supporting or excluding various scenarios for the theory of everything (TOE). For example, some versions of superstring theory predict deviations from Newton's gravitational potential on length scales far above the Planck scale. However, no such deviations could so far be found.

The breaking of Lorentz symmetry between space and time generated by the $c^{−1}\rightarrow 0$ limit also inspires research on a purely theoretical level. For example, in 2009 Petr Hořava proposed a new theory of gravity that modifies general relativity on small length scales. This has already resulted in more than \num{2600} subsequent publications. Nonetheless, the status of this so-called Hořava-Lifshitz Gravity Theory remains unclear.
