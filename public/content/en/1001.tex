
\newpage \addsec{Statistical mechanics and Newtonian gravity $(G,0,0,\kboltz)$}
\label{sec:1001}

The union $(G,0,0,\kboltz)$ of Newton's gravity theory $(G,0,0,0)$ and Boltzmann's statistical mechanics $(0,0,0,\kboltz)$ causes neither technical nor conceptual problems. It has many interesting applications, for which we give two examples below: The barometric height formula for the atmospheric pressure in planetary bodies such as our Earth, and a rough picture of star formation from contracting gaseous clouds. However, it should be pointed out that this union does not lead to more than the \emph{sum of its parts}, i.\,e.\ one does not find an interesting, new \emph{unified theory}.


\subsection*{Barometric height formula}

An ideal gas of molecules of mass $m$ in a gravitational field generated by a mass $M$ causes the barometric gas pressure to decline exponentially with the ratio of the gravitational energy of the gas molecules and their thermal energy. To derive it, we may simply consider the Boltzmann distribution of molecules in the atmosphere. A molecule of mass $m$ situated at height $H$ has a potential energy $E=m g H$, where the gravitational acceleration $g$ on and slightly above the surface of the earth is expressed through Newton's gravitational constant $G$, the earth's mass $M$ and radius $R$ as $g=G M/R^2$. This yields the result that the molecular probability distribution is proportional to 

\begin{equation*}\label{eq:barometric_height}
  \exp\left(-\frac{m M}{R^2}\frac{G}{\kboltz} \frac{H}{T}\right).
\end{equation*}
%
Note the appearance of the ratio $G/\kboltz$ of the two relevant fundamental constants.

We should distinguish these from the situational (random) constants $m, M, R^2$. The variables of this problem are $H$ and $T$. The exact probability distribution is given by

\begin{equation*}\label{eq:barometric_height_full_1}
  P(H)=\frac{m M}{R^2}\frac{G}{\kboltz T} \exp\left(-\frac{m M}{R^2}\frac{G}{\kboltz} \frac{H}{T}\right),
\end{equation*} 
%
which indeed satisfies $\int_0^\infty dH P(H)=1$. It follows that the ratio of barometric pressures $p(H_2)$, $p(H_1)$ at heights $H_2$, $H_1$ is given by

\begin{equation*}\label{eq:barometric_height_full_2}
  \frac{p(H_2)}{p(H_1)}=\exp\left(-\frac{m M}{R^2}\frac{G}{\kboltz} \frac{H_2-H_1}{T}\right).
\end{equation*}


\subsection*{Star formation}

What is the critical mass, where interstellar gas clouds collapse and form a new star? The Jeans instability, named after Sir James Jeans, describes the limit above which the outward pointing internal gas pressure is no longer strong enough to balance out the inward pointing gravitational pull of the ensemble of particles in the gas. The gas cloud undergoes gravitational collapse, whereby the first stage in the formation of a protostar is reached.

This problem, of crucial importance in astrophysics, may be approximately treated by applying the basic principles of Newtonian gravity in conjunction with the basics principles of classical statistical mechanics. Of course, the total mass involved should not be too large, and the quantum effects of compressing the gas should be neglected. One then finds, using a variety of further approximations and simplifying assumptions, the critical \emph{Jeans mass}, above which the gravitational potential energy of the system is greater than the kinetic energy of its components, leading to the gravitational collapse needed for a new star to be born:

\begin{equation*}
  M_{\mathrm {Jeans}}=\alpha\,
    \left(\frac {\kboltz}{G}\right)^{\frac{3}{2}}\,
    \sqrt {{\frac {1}{\rho}}\, 
    \left({\frac {T}{m}}\right)^3}\,.
\end{equation*}
%
Here, $T$ is the absolute temperature of the gas, $m$ is the mass of its elementary constituents and $\rho$ is their density, while $\alpha$ is a numerical constant that depends on the details of the approximation (one typically finds a number between 1 and 10). Most important for us is the dependence of $M_{\mathrm {Jeans}}$ on the two fundamental constants $\kboltz$ and $G$. Note that $c^{-1}$ and $h$ do not appear, since we disregarded, respectively, all relativistic and all quantum effects.

